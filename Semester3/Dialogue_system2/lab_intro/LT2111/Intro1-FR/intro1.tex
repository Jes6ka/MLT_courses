\documentclass[a4,landscape]{seminar}
\usepackage[utf8x]{inputenc}
\usepackage[pdftex]{graphicx}
\usepackage{fancybox}
\usepackage{url}
% \usepackage{sem-a4}

\pdfcompresslevel=9
\pdfpageheight=8.27truein
\pdfpagewidth=11.69truein
\pdfhorigin=1truein
\pdfvorigin=1truein
\slideheight 14truein
\slidewidth 22truein

\newpagestyle{roberts}
        {\thepage\hfil Introduction to the Language Technology Lab autumn 2017\hfil\hfil\hfil Fei Roth \texttt{fei.roth@gu.se}\hfil}{}
\pagestyle{roberts}
\slideframe{oval}

\begin{document}
% === Slide 1 ===
\begin{slide}
  {\Large Contents}
  \begin{itemize}
  \item General Introduction to the Computer Lab Environment
  \item macOS Sierra Orientation
    \begin{itemize}
    \item Desktop environment, File Handling, Applications, and Keyboard
    \end{itemize}
  \item Terminal and the Command-Line Interface, \emph{CLI}
  \item Emacs, \emph{the} text editor, Lisp interpreter, compiler, debugger, ...
  \end{itemize}
\end{slide}
% === Slide 2 ===
\begin{slide}
  {\Large Computer Lab Environment}
  \begin{itemize}
  \item Access to the Lab
    \begin{itemize}
    \item Access to the lab is granted Monday-Sunday between 07:30 am to
     09:00 pm. The room should be empty by 09:30 pm, when the alarm is
      automatically switched on. The computer lab is closed on legal
      holidays. Please report lost GU keycards to the FLoV
      administration as quickly as possible.
    \end{itemize}
  \item Computer Environment
    \begin{itemize}
    \item 15 Apple iMac workstations for students, 1 workstation for
      the teacher
    \item 24 seats within workstation reach, where 6 computers can only
      be used by one student (some extra chairs are available in the room)
    \item 0 printers (multifunction devices and network printers)
      \begin{itemize}
      \item Different types of printers can be found in all university buildings.
      \item Students can buy printer quota at the service
        centers. Please observe that every print-out and copy is charged. Color
        prints are more expensive.
      \item It is recommended to check the printing settings before printing.
      \item PCClient (software) and GUprint (print and copy service)
      \end{itemize}
    \end{itemize}
  \end{itemize}
\end{slide}
% === Slide 3 ===
\begin{slide}
  {\Large Computer Lab Environment (cont.)}
  \begin{itemize}
  \item Policies and rules
    \begin{itemize}
    \item Please observe that the common regulations for the use of
      the university’s computer network apply.
    \item For everybody's convenience, please do not eat or drink in
      the computer lab. For coffee breaks, lunch, etc. you can use the
      student lounges which are available in this building
      (Eklandagatan 86).
    \item Non-study activities like private gaming are not allowed in
      the computer lab.
    \item Class activities in the lab always have priority. Apart from
      that you are encouraged to use the lab for assignment work etc. as
      much as possible.
    \item WiFi (Eduroam and GoteborgsUniversitet) as well as extra
      power sockets are available in the computer lab, in case you
      need to use your private computer alongside the lab computers.
    \end{itemize}
  \end{itemize}
\end{slide}
% === Slide 4 ===
\begin{slide}
  {\Large Apple macOS Sierra}
  \begin{itemize}
  \item Accounts and Login
    \begin{itemize}
    \item To be able to login, please use your GU student account, \texttt{gus•••••}.
    \item There is no disk quota for single users. Please use the
      available storage space with
      care and do not store large files and data unnecessarily.
    \item Copyrighted material obtained illegally must NOT be stored
      on the computers.
    \item Your student account will be deleted, if you do not register
      for the MLT program's courses for more than two consecutive semesters.
    \end{itemize}
  \item Apple Menu\\[1ex]
    Where you manage miscellaneous features
    \begin{itemize}
    \item Logging out, pls logout after a session as you can not lock the screen.
    \item System Preferences, e.g.
      \begin{itemize}
      \item Audio I/O settings, pls use these and do not unplug the headsets.
      \item Language settings (...)
      \end{itemize}
    \end{itemize}
  \end{itemize}
  \clearpage{}
  \begin{itemize}
  \item Keyboard
    \begin{itemize}
    \item The command key, \texttt{cmd}, used for common shortcuts instead of % , ⌘, 
      ctrl on a PC.
    \item Common character can be hard to find as they are not printed
      on the keyboard, notably \texttt{\{}, \texttt{\}}, \texttt{[}, \texttt{]}, \texttt{|}, \textbackslash, and \textasciitilde. Open the keyboard viewer to get help finding the keys.
    \end{itemize}
  \item Finder\\[1ex]
    Where you do your main filesystem management, launching applications, etc. Works almost like Windows Explorer.
    \begin{itemize}
    \item Local folders\\[1ex]
      Folders on your account that is stored on the lab computer
      currently in use. The contents of these folders can be cleaned
      at any time, and will not be available if the user log in to
      another lab computer. The local folders are
      \clearpage{}
      \begin{itemize}
      \item Downloads,
      \item Dropbox,
      \item Movies,
      \item Music, and
      \item Pictures.
      \end{itemize}
    \item Documents folder\\[1ex]
      Place all your important files/folders in this folder as it is
      guaranteed to be backed up. Saving files on the Desktop is not
      safe!
    \item Get info, \texttt{cmd-I}\\[1ex]
      In the info window you can change permissions for the
      file/folder and change default program for opening these kind of
      files.
    \end{itemize}
    \clearpage{}
  \item Dock\\[1ex]
    The dock is divided into two part, on the right-hand-side you have
    shortcuts to folders, minimised windows, and the trash can. On the
    left-hand-side you have shortcuts to your applications.\\\\
    Add folders and applications by draging them from a Finder window.
    Remove folders and applications from the dock by draging them from
    the Dock..
  \item Backups\\[1ex]
    Again -- only the Documents folder is guaranteed to be backed up!
    The backups are stored in a hidden folder, \texttt{.snapshot}, and
    it can be viewed in Finder by
    \begin{itemize}
    \item opening the folder where the lost file was,
    \item clicking \texttt{Go to folder\dots} in the Finder's \texttt{Go} menu\\[1ex]
      Go → Go to folder\dots
    \item write \texttt{.snapshot} in the pop-up window, and hit \texttt{<enter>}.
    \end{itemize}
    \clearpage{}
  \item Support\\[1ex]
    Report problems in the lab to your teacher or Fei Roth (room T224,
    Olof Wijksgatan 6, mobile phone: 0766186784). Also, if possible, email problems to\\[1ex]
    \texttt{fei.roth@gu.se}\\\\
    We do not support your private computers.
  \end{itemize}
\end{slide}
\begin{slide} 
  {\Large Terminal -- Command-Line Interface, \emph{CLI}}\\[1ex]
  In the terminal application you have a text shell, \emph{bash}, where you
  can enter commands, aka the command-line interface.
  \begin{itemize}
  \item Commands and navigating the commandline
    \begin{itemize}
    \item The prompt, where you write your commands\\[1ex]
      \texttt{COMPUTERNAME:CWD USERNAME\$}
    \item Command format\\[1ex]
      \texttt{COMMANDNAME ARG1 ARG2 ARG3 \dots}\\[1ex]
      \texttt{ARG}s can be optional, \texttt{ARG}s can eg be options
      (aka switches/flags), filenames, or \dots\\[1ex]
      \texttt{ls -l filename.txt}
      \clearpage{}
    \item Editing and navigating the commandline
      \begin{itemize}
      \item After finishing a commandline, hit \texttt{<enter>} to execute
      \item Keyboard shortcuts for editing on the command line\\[1ex]
        \begin{tabular}{|r|l|}
          \hline
          \multicolumn{2}{|l|}{Editing, \texttt{C-} = ctrl key }\\\hline\hline
          \texttt{→} or \texttt{C-f} & forward character\\
          \texttt{←} or \texttt{C-b} & back character\\
          \texttt{C-e} & go to end of line\\
          \texttt{C-a} & go to beginning of line\\
          \texttt{C-d} & delete character\\
          \texttt{C-k} & kill all following char\\
          \texttt{C-u} & discard line\\
          \texttt{C-w} & remove last word typed\\
          \texttt{C-l} & clear screen\\\hline
          \end{tabular}
          \begin{tabular}{|r|l|}
          \hline
          \multicolumn{2}{|l|}{History, \texttt{C-} = ctrl key }\\\hline\hline
          \texttt{↓} or \texttt{C-n} & next line\\
          \texttt{↑} or \texttt{C-p} & previous line\\\hline
          \multicolumn{2}{l}{}\\\hline
          \multicolumn{2}{|l|}{Misc, \texttt{C-} = ctrl key }\\\hline\hline
          \texttt{C-c} & abort current command\\
          \texttt{<TAB>} & tab-completion\\\hline
        \end{tabular}
      \end{itemize}
      \clearpage{}
    \item Man pages -- online manual in the terminal\\[1ex]
      To read a man page for some command, issue the command\\[1ex]
      \texttt{man \textsl{COMMAND}}\\[1ex]
      Eg \texttt{man man} for reading the man page for the man command. A man page can contain numerous sections, a few of interest are
      \begin{itemize}
      \item \texttt{NAME}, name of comand followed by a short description
      \item \texttt{SYNOPSIS}, how to use and what options and args are expected
      \item \texttt{DESCRIPTION}, longer description of the command
      \item \texttt{OPTIONS}, what an option does (sometimes in the description section)
      \item \texttt{EXAMPLES}, examples usages
      \end{itemize}
      \clearpage{}
      Navigating the man pages, default keys
        \begin{tabular}{|r|l|}
          \hline
          $<$SPACE$>$ or $<$PgDn$>$ & page down\\
          b or $<$PgUp$>$ & page up\\
          q & exit the man page (takes you back to bash)\\\hline
        \end{tabular}
        \\\\
        Eg. the command \texttt{say} will convert text to speech and output it the built-in MacOS X
        tts engine\\[1ex]
        \texttt{say Hello, MLT-students}\\[1ex]
        using the man for says\\[1ex]
        \texttt{man say}\\[1ex]
        one can eg find out how to use another voice\\[1ex]
        \texttt{say -v Vicki Hello, masterstudents}\\
        \texttt{say --voice=Vicki Hello, master students}
    \end{itemize}
    \clearpage{}
  \item File management in the shell
    \begin{itemize}
    \item Filenames\\[1ex]
      \texttt{NAME.EXT}\\[1ex]
      where \texttt{NAME} is the basename and \texttt{.EXT} indicates
      the content type of the file, eg \texttt{.txt}, \texttt{.py} etc.\\\\
      N.B. file and folder names are, per default, case insensitive on MacOS X, eg \texttt{FileName.TXT} = \texttt{filename.txt} when in the same folder
    \item The filetree\\[1ex]
      Viewing the filesystem as a reversed tree, root at top. TODO: graph \\\\
      N.B. a folder is now known as a \emph{directory} when working in the terminal!
      \clearpage{}
    \item Pathnames, Absolute vs Relative\\[1ex]
      A path is used to specify a unique location in a filesystem. The path point to a location by following the filetree using a slash, \texttt{/}, to delimit each item, folder or file in the tree.\\\\
      An absolute path reference includes all levels from root, eg:\\[1ex]
      \texttt{/Users/gus•••••/Documents/file.txt}\\
      \texttt{/opt/mlt/courses/labintro/}\\[1ex]
      A relative path reference points to a file from the current working directory, eg\\if the current working directory is \texttt{/Users/gus•••••} to the same file, \\ \texttt{file.txt}, above with:\\[1ex]
      \texttt{Documents/file.txt}\\
      \texttt{./Documents/file.txt}\\
      \texttt{../gus•••••/Documents/file.txt}\\[1ex]
      \texttt{..} denotes on level up and\\
      \texttt{.} denotes this level
    \item Shortcuts and wildcards\\[1ex]
      Shortcuts and variables may be expanded to paths, eg the following also points to the above file, \texttt{file.txt}\\[1ex]
      \texttt{\textasciitilde/Documents/file.txt}\\
      \texttt{\$HOME/Documents/file.txt}\\[1ex]
      as both the shortcut \texttt{\textasciitilde}~and the variable \texttt{\$HOME} expands to \texttt{/Users/gus•••••}\\\\
      With the wildcard \texttt{*} you can match any number of unknown characters in eg\\[1ex]
      \texttt{\textasciitilde/*/file.txt}\\[1ex]
      points to the same file as above as \texttt{*} here matches Documents. The path\\[1ex]
      \texttt{/Users/gus•••••/Documents/*.txt}\\[1ex]
      points to all files in the Documents directory ending with \texttt{.txt}.
      \clearpage{}
    \item Listing files and directories\\[1ex]
      \begin{tabular}{|l|l|}
        \hline
        \texttt{ls} & list contents in directory\\
        \texttt{ls -l} & list contents in directory with long format\\
        \texttt{ls -l PATH} & list \texttt{PATH} with long format\\\hline
      \end{tabular}
    \item Walking the filetree\\[1ex]
      \begin{tabular}{|l|l|}
        \hline
        \texttt{pwd} & print current working direcotry\\
        \texttt{cd} & change working directory to home directory, \textasciitilde\\
        \texttt{cd PATH} & change working diretory to \texttt{PATH}\\\hline
      \end{tabular}
    \item Create and remove directories\\[1ex]
      \begin{tabular}{|l|l|}
        \hline
        \texttt{mkdir PATH} & make directory \texttt{PATH}\\
        \texttt{rmdir PATH} & remove directory \texttt{PATH}\\\hline
      \end{tabular}
      \clearpage{}
    \item Copy, move, and remove files
      \begin{tabular}{|l|l|}
        \hline
        \texttt{cp PATH1 PATH2} & copy \texttt{PATH1} to \texttt{PATH2}\\
        \texttt{cp -r PATH1 PATH2} & recursive copy of \texttt{PATH1} to \texttt{PATH2}\\
        \texttt{mv PATH1 PATH2} & move, or rename, \texttt{PATH1} to \texttt{PATH2}\\
        \texttt{rm PATH1} & remove \texttt{PATH1}\\\hline
      \end{tabular}\\\\
      Use \texttt{.} as \texttt{PATH2} with \texttt{cp} to copy files to the current working directory without changing its name, eg\\[1ex]
      \texttt{cp /opt/mlt/courses/labintro/testing.txt .}\\[1ex]
      will copy the file \texttt{testing.txt} here.
    \item Misc commands\\[1ex]
      \begin{tabular}{|l|l|}
        \hline
        \texttt{file PATH} & determine file type, character encoding etc\\
        \texttt{du}   & display disk usage\\
        \texttt{open PATH} & open files and directories\\\hline
      \end{tabular}
    \end{itemize}
    \clearpage{}
  \item Shells in shells\\[1ex]
    Sometimes you will have to start a program that uses its own text
    shell, having its own set of commands, e.g.
    \begin{itemize}
    \item the Python interpreter shell, and
    \item the XFST interpreter shell etc.
    \end{itemize}
    Note the initial text and that the prompt will probably be
    different, though navigating and editing text in these new
    ``shells'' is often the same as described for bash above.

    To exit a shell you can try \texttt{C-d} on a empty line, or
    execute the \texttt{exit} command.
  \end{itemize}
\end{slide}
\begin{slide}
  {\Large References}\\[1ex]
  \url{http://en.wikibooks.org/wiki/Linux_Guide/Using_the_shell}
  \url{http://en.wikibooks.org/wiki/Linux_Guide/Linux_commands}
  \url{https://www.linuxtrainingacademy.com/}
\end{slide}
\end{document}

%%% Local Variables:
%%% mode: latex
%%% TeX-master: t
%%% End:
